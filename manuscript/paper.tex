\documentclass[12pt]{article}

%% preamble: Keep it clean; only include those you need
\usepackage{amsmath}
\usepackage[margin = 1in]{geometry}
\usepackage{graphicx}
\usepackage{booktabs}
\usepackage{natbib}

% highlighting hyper links
\usepackage[colorlinks=true, citecolor=blue]{hyperref}


%% meta data

\title{The Effects of the 2023 Rule Changes in Major League Baseball}
\author{Luke Noel\\
  Department of Statistics\\
  University of Connecticut
}

\begin{document}
\maketitle

\begin{abstract}
Major League Baseball (MLB) is a sport continuously evolving through rule
changes dedicated to improving game dynamics, player safety, and fan engagement.
Prior to the start of the 2023 season, MLB's commissioner Rob Manfred
implemented a set of significant rule changes, headlined by the introduction of 
a pitch clock, defensive shift restrictions, and bigger bases. Although these
changes were intended to improve pace of play, action, and safety of the game,
their actual impact on various aspects of baseball performance and strategy is 
not well understood. This study uses statistical methods to evaluate the effects
of the rule changes on game duration, scoring, hitting, pitching, fielding, and
baserunning. We compare the 2023 season data with the previous seasons using
descriptive statistics, hypothesis testing, and regression analysis. INSERT
major findings here.
\end{abstract}


\section{Introduction}
\label{sec:intro}

Major League Baseball (MLB), often referred to as America's pastime, is a sport
rich in tradition and innovation. Throughout its long history, MLB has enacted
numerous rule changes aimed at improving the game's relevance, competitiveness,
and entertainment value. For instance, in 1973, the designated hitter rule was
implemented for all American League teams, with the hopes of increasing
offensive production as a whole. However, we are not really interested in ``why''
the rules are changed, but more the underlying effects of the change(s). This
can be seen in \citet{Cooley} where the effects of the designated hitter rule
on player longevity, pitcher complete games, and number of hit batters are
analyzed. We see many more major and minor MLB rule changes come into action
throughout the course of its history, like the recent ``ghost runner'' rule
carried out in the weird, shortened, COVID-19 season in 2020, where a baserunner 
automatically starts out on second-base in extra-innings.

Building upon this willingness for change, prior to the start of the 2023 season,
Rob Manfred, MLB's commissioner, implemented a set of significant rule changes
intended to improve the pace of play, action, and safety of the game. Here is
a list of the new rules and a short description for each, according to \citet{Castrovince}:

\begin{itemize}
  \item \textbf{Pitch Timer:} 30-second timer between batters. Between pitches,
  a 15-second timer with the bases empty and a 20-second timer with runners on base.
  Pitchers who violate the timer are charged with an automatic ball.
  Batters who violate the timer are charged with an automatic strike.

  \item \textbf{Defensive Shift Limits:} The defensive team must have a minimum
  of four players on the infield, with at least two infielders completely on either
  side of second base.

  \item \textbf{Bigger Bases:} The bases, which traditionally have been 15 inches
  square, will instead be 18 inches square.

  \item \textbf{Pitcher Disengagement Limits:} Pitchers are limited to two
  disengagements (pickoff attempts or step-offs) per plate appearance. If a third
  pickoff attempt is made, the runner automatically advances one base if the 
  pickoff attempt is not successful.
\end{itemize}

These rule changes may seem ridiculous to many baseball ``purists'' out there,
but here is a quote from Commissioner Manfred that explains the rationale:

\begin{quote}
  We've tried to address the concerns expressed in a thoughtful way, respectful,
  always, of the history and traditions of the game, and of player concerns. Our
  guiding star in thinking about changes to the game has always been our fans.
  What do our fans want to see on the field? We've conducted thorough and ongoing
  research with our fans, and certain things are really clear. Number 1, fans want
  games with better pace. Two, fans want more action, more balls in play. And three,
  fans want to see more of the athleticism of our great players.
\end{quote}

Also, keep in mind that these rule changes have been thoroughly tested and refined
in the Minor Leagues for many years so it's not like it was a split decision to
implement them in the Majors. Each of the rules were actually tested in about 8,000
Minor League games \citep{Castrovince}.

The next section is a Literature Review that details some relevant works on the
topic, but it is important to know that the data is so new that the impact of the
rule changes on various aspects of baseball performance metrics and strategies 
is not well understood yet. This study uses statistical methods to evaluate the
effects of the rule changes on game duration, scoring, hitting, pitching, fielding,
and baserunning. For instance, a major baseball statistic that is analyzed is
stolen bases (SB) which we should see an increase of across the league due to the
rule changes. Bigger bases mean slightly less distance traveled to get the
SB, and the pitch timer means that baserunners can time up the pitchers easier
and know when to start stealing. We compare the 2023 season data with the previous
seasons using descriptive statistics, hypothesis testing, and regression analysis.

This analysis is important as it holds the potential to inform future rule changes,
offering empirical insights that can guide decision-makers in shaping the game's
future. The data and findings generated by this research may serve as a valuable
resource for MLB executives, team managers, and players in adapting to the new
rule landscape. Equally crucial, fans and enthusiasts will gain deeper insights
into the impact of these changes on the sport they love, providing context to
their experiences as spectators.

% roadmap
The rest of the paper is organized as follows.
A literature review of relevant works will be examined in Section~\ref{sec:lit}.
The data used in the study will be presented in Section~\ref{sec:data}.
The methods are described in Section~\ref{sec:meth}.
The results are reported in Section~\ref{sec:resu}.
A discussion concludes in Section~\ref{sec:disc}.


\section{Literature Review}
\label{sec:lit}

As stated previously, there has been no scholarly literature yet on the effects
of the 2023 MLB rule changes (as of fall 2023), since it is so new. However,
there has been research done already on how a certain rule change/changes
affected a sport. For example, as referenced above, \citet{Cooley} investigated
the impact of the Designated Hitter (DH) in the MLB, a rule change that let a batter
hit for a pitcher in the lineup, instituted in 1973. They analyzed how the
implementation of the DH impacted certain metrics like attendance, complete games
by pitchers (CG), and total runs scored. They did so using various statistical
methods such as correlation analysis, regression (particularly r-squared analysis),
and compound average growth rates. They concluded that the DH led to more offensive
production, a decrease in total complete games by pitchers, and had no significant
effect on overall fan attendance at games.

However, we do not need to just look at Major League Baseball to see
the effects of some rule changes in sports. For instance, \citet{Madera}
discusses the impact of certain rule changes on game-related statistics in
professional men's water polo matches. They did so by calculating means and
standard deviations for certain game-related stats such as possession time,
total goals, etc., and running one-way ANOVA to test for significance in the
difference of these metrics pre and post rule changes. In a similar fashion,
\citet{Nourayi} analyzed the effects of the new defensive 3-second rule and
elimination of hand checking in the National Basketball Association (NBA), rule
changes aimed at improving the flow and pace of the game. This was done using
many different t-tests which resulted in finding that the total FGA (Field Goals
Attempted) and 3FGA (3-Point Field Goals Attempted) were significantly greater
in the post rule change period.

Also, we do not need to just look at rule changes specifically, we can explore
a ``before and after analysis'' of some other change made in sports. \citet{Price}
is a good illustration of this is, as they conducted an analysis of the COVID
``bubble'' in the NBA playoffs and how this affected home-court advantage.
Nine different z-tests were conducted to try to see if there was significance in
the difference in free-throw rate, winning percentage, etc. in the 2020 season
compared to previous years. This highly relates to our study, but we will instead
be testing the difference in say stolen bases for example, from the 2023 season
compared to previous years. \citet{Zhang} is another great example as they
focused on the effect of the video assistant referee (VAR) on referee's decisions
in the FIFA Women's World Cup. They used a Mann-Whitney U test to compare differences
in game-related variables such as playing time, penalties, and goals, with and
without VAR.


\section{Data}
\label{sec:data}

The data used for this analysis was collected from \href{https://baseballsavant.mlb.com/}{Baseball Savant},
a comprehensive and interactive platform that contains advanced statistics and
data related to Major League Baseball (MLB). The main variables of interest
are Stolen Bases (SB), Batting average (BA), Strike Outs (SO), and Batting
Average on Balls in Play (BABIP), among some others. These are game-related
baseball statistics that will be analyzed similar to those used in the literature
above, like we saw with Free Throw Rate in the NBA, for example. These variables
were selected based on domain knowledge and intuition about what metrics may be
affected most by the new MLB rule changes. We expect to see Stolen Bases increase
across the league, due to the pitch clock and pitcher disengagement limits. Batting
Average, Batting Average on Balls in Play, and Slugging (SLG) may be affected by
the limit of the defensive shift, as this rule change may increase the likelihood
of a hit. In a similar vein, Strike Outs and Walks (BB) may see an impact as both
pitchers and batters adjust to the pitch clock. Other variables included in the
study:
\begin{itemize}
  \item Ground Into Double Play (GIDP)
  \item Hits on Ground Balls (Hit-GB)
  \item Expected Batting Average (xBA)
\end{itemize}
We expect these variables will be affected by the limit of the defensive shift.
Also, pace of play (game time) data was collected from \href{https://www.baseball-reference.com/}{Baseball Reference}.
Each row of this dataset represents a single season from 2023 dating back to 1998
(26 observations total), containing the Average Game Time (in hours) for that season.
Intuitively, we should expect this variable to have a significant decrease for the
2023 season, due to the introduction of the pitch clock making the games go faster.
Ideally, we would have used datasets containing the game times for \underline{each}
game in both seasons, however this was not publicly available.

The main data used though was collected using the Custom Leaderboard feature
on Baseball Savant, where each variable above was added to the dataset, with each
row representing a single player's overall statistics. The data was then filtered
to only include the 2023 season, and only include players with a minimum 100 Plate Appearances (PA)
that season to limit outliers (100 min. PA is a popular threshold used for 
baseball player analysis). This gives us 461 observations for the 2023 season,
again with each row being a unique batter. The same steps were then applied to
create a comparison dataset now filtering for the 2022 season, which has 469
observations total. This gives us a baseline to compare the variables in the
season where the rule changes were in effect (2023), to the previous season where
they were not (2022).

Different data formats were considered, such as using team-wide statistics
for each season instead of player-wide, where the dataset would be 30 rows (one
for each team). Using a combination of previous seasons data for comparison (not
just 2022) was considered also, but we settled on the data we used as it seemed
the most straightforward for our analysis and limited potential confounding
factors. Data from complete game logs for each game of the 2023 season would have
been an option, however this data is not yet publicly available. More on these topics
are discussed in Section's~\ref{sec:meth} and \ref{sec:disc}.


\section{Methods}
\label{sec:meth}

Use this section to present the methodologies that will generate results by
analyzing the data.


\section{Results}
\label{sec:resu}

Summarize the results.


\section{Discussion}
\label{sec:disc}

What are the main contributions again?

What are the limitations of this study?

What are worth pursuing further in the future?


\bibliographystyle{chicago}
\bibliography{refs.bib}


\end{document}