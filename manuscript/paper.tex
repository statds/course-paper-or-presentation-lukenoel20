\documentclass[12pt]{article}

%% preamble: Keep it clean; only include those you need
\usepackage{amsmath}
\usepackage[margin = 1in]{geometry}
\usepackage{graphicx}
\usepackage{booktabs}
\usepackage{natbib}

% highlighting hyper links
\usepackage[colorlinks=true, citecolor=blue]{hyperref}


%% meta data

\title{The Effects of the 2023 Rule Changes in Major League Baseball}
\author{Luke Noel\\
  Department of Statistics\\
  University of Connecticut
}

\begin{document}
\maketitle

\begin{abstract}
Major League Baseball (MLB) is a sport continuously evolving through rule
changes dedicated to improving game dynamics, player safety, and fan engagement.
Prior to the start of the 2023 season, MLB's commissioner Rob Manfred
implemented a set of significant rule changes, headlined by the introduction of 
a pitch clock, defensive shift restrictions, and bigger bases. Although these
changes were intended to improve pace of play, action, and safety of the game,
their actual impact on various aspects of baseball performance and strategy is 
not well understood. This study uses statistical methods to evaluate the effects
of the rule changes on game duration, scoring, hitting, pitching, fielding, and
baserunning. We compare the 2023 season data with the previous seasons using
descriptive statistics, hypothesis testing, and regression analysis. INSERT
major findings here.
\end{abstract}


\section{Introduction}
\label{sec:intro}

Major League Baseball (MLB), often referred to as America's pastime, is a sport
rich in tradition and innovation. Throughout its long history, MLB has enacted
numerous rule changes aimed at improving the game's relevance, competitiveness,
and entertainment value. For instance, in 1973, the designated hitter rule was
implemented for all American League teams, with the hopes of increasing
offensive production as a whole. However, we are not really interested in \emph{why}
the rules are changed, but more the underlying effects of the change(s). This
can be seen in \citet{Cooley} where the effects of the designated hitter rule
on player longevity, pitcher complete games, and number of hit batters are
analyzed. We see many more major and minor MLB rule changes come into action
throughout the course of its history, like the recent \emph{ghost runner} rule
carried out in the weird, shortened, COVID-19 season in 2020, where a baserunner 
automatically starts out on second-base in extra-innings.

Building upon this willingness for change, prior to the start of the 2023 season,
Rob Manfred, MLB's commissioner, implemented a set of significant rule changes
intended to improve the pace of play, action, and safety of the game. Here is
a list of the new rules and a short description for each, according to \citet{Castrovince}:

\begin{itemize}
  \item \textbf{Pitch Timer:} 30-second timer between batters. Between pitches,
  a 15-second timer with the bases empty and a 20-second timer with runners on base.
  Pitchers who violate the timer are charged with an automatic ball.
  Batters who violate the timer are charged with an automatic strike.

  \item \textbf{Defensive Shift Limits:} The defensive team must have a minimum
  of four players on the infield, with at least two infielders completely on either
  side of second base.

  \item \textbf{Bigger Bases:} The bases, which traditionally have been 15 inches
  square, will instead be 18 inches square.

  \item \textbf{Pitcher Disengagement Limits:} Pitchers are limited to two
  disengagements (pickoff attempts or step-offs) per plate appearance. If a third
  pickoff attempt is made, the runner automatically advances one base if the 
  pickoff attempt is not successful.
\end{itemize}







\bibliographystyle{chicago}
\bibliography{refs.bib}


\end{document}