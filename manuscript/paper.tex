\documentclass[12pt]{article}

%% preamble: Keep it clean; only include those you need
\usepackage{amsmath}
\usepackage[margin = 1in]{geometry}
\usepackage{graphicx}
\usepackage{booktabs}
\usepackage{natbib}

% highlighting hyper links
\usepackage[colorlinks=true, citecolor=blue]{hyperref}


%% meta data

\title{The Effects of the 2023 Rule Changes in Major League Baseball}
\author{Luke Noel\\
  Department of Statistics\\
  University of Connecticut
}

\begin{document}
\maketitle

\begin{abstract}
Major League Baseball (MLB) is a sport continuously evolving through rule
changes dedicated to improving game dynamics, player safety, and fan engagement.
Prior to the start of the 2023 season, MLB's commissioner Rob Manfred
implemented a set of significant rule changes, headlined by the introduction of 
a pitch clock, defensive shift restrictions, and bigger bases. Although these
changes were intended to improve pace of play, action, and safety of the game,
their actual impact on various aspects of baseball performance and strategy is 
not well understood. This study uses statistical methods to evaluate the effects
of the rule changes on game duration, scoring, hitting, pitching, fielding, and
baserunning. We compare the 2023 season data with the previous seasons using
descriptive statistics, hypothesis testing, and regression analysis. INSERT
major findings here.
\end{abstract}


\section{Introduction}
\label{sec:intro}

Major League Baseball (MLB), often referred to as America's pastime, is a sport
rich in tradition and innovation. Throughout its long history, MLB has enacted
numerous rule changes aimed at improving the game's relevance, competitiveness,
and entertainment value. For instance, in 1973, the designated hitter rule was
implemented for all American League teams, with the hopes of increasing
offensive production as a whole. However, we are not really interested in "why"
the rules are changed, but more the underlying effects of the change(s). This
can be seen in \citet{Cooley} where the effects of the designated hitter rule
on player longevity, pitcher complete games, and number of hit batters are
analyzed.






\end{document}