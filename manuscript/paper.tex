\documentclass[12pt]{article}

%% preamble:
\usepackage{amsmath}
\usepackage[margin = 1in]{geometry}
\usepackage{graphicx}
\usepackage{booktabs}
\usepackage{natbib}
\usepackage{setspace}

% highlighting hyper links
\usepackage[colorlinks=true, citecolor=blue]{hyperref}

\setlength{\tabcolsep}{10pt}
\doublespacing

%% meta data

\title{The Effects of the 2023 Rule Changes in Major League Baseball}
\author{Luke Noel\\
  Department of Statistics\\
  University of Connecticut
}

\begin{document}
\maketitle

\begin{abstract}
Major League Baseball (MLB) is a sport continuously evolving through rule
changes dedicated to improving game dynamics, player safety, and fan engagement.
Prior to the start of the 2023 season, MLB's commissioner Rob Manfred
implemented a set of significant rule changes, headlined by the introduction of 
a pitch clock, defensive shift restrictions, and bigger bases. Although these
changes were intended to improve pace of play, action, and safety of the game,
their actual impact on various aspects of baseball performance and strategy is 
not well understood. This study uses statistical methods to evaluate the effects
of the rule changes on game duration, scoring, hitting, pitching, fielding, and
baserunning. We compare the 2023 season data with the previous seasons using
descriptive statistics, hypothesis testing, and regression analysis. INSERT
major findings here.
\end{abstract}


\section{Introduction}
\label{sec:intro}

Major League Baseball (MLB), often referred to as America's pastime, is a sport
rich in tradition and innovation. Throughout its long history, MLB has enacted
numerous rule changes aimed at improving the game's relevance, competitiveness,
and entertainment value. For instance, in 1973, the designated hitter rule was
implemented for all American League teams, with the hopes of increasing
offensive production as a whole. However, we are not really interested in ``why''
the rules are changed, but more the underlying effects of the change(s). This
can be seen in \citet{Cooley} where the effects of the designated hitter rule
on player longevity, pitcher complete games, and number of hit batters are
analyzed. We see many more major and minor MLB rule changes come into action
throughout the course of its history, like the recent ``ghost runner'' rule
carried out in the weird, shortened, COVID-19 season in 2020, where a baserunner 
automatically starts out on second-base in extra-innings.

Building upon this willingness for change, prior to the start of the 2023 season,
Rob Manfred, MLB's commissioner, implemented a set of significant rule changes
intended to improve the pace of play, action, and safety of the game. Here is
a list of the new rules and a short description for each, according to \citet{Castrovince}:

\begin{itemize}
  \item \textbf{Pitch Timer:} 30-second timer between batters. Between pitches,
  a 15-second timer with the bases empty and a 20-second timer with runners on base.
  Pitchers who violate the timer are charged with an automatic ball.
  Batters who violate the timer are charged with an automatic strike.

  \item \textbf{Defensive Shift Limits:} The defensive team must have a minimum
  of four players on the infield, with at least two infielders completely on either
  side of second base.

  \item \textbf{Bigger Bases:} The bases, which traditionally have been 15 inches
  square, will instead be 18 inches square.

  \item \textbf{Pitcher Disengagement Limits:} Pitchers are limited to two
  disengagements (pickoff attempts or step-offs) per plate appearance. If a third
  pickoff attempt is made, the runner automatically advances one base if the 
  pickoff attempt is not successful.
\end{itemize}

These rule changes may seem ridiculous to many baseball ``purists'' out there,
but here is a quote from Commissioner Manfred that explains the rationale:

\begin{quote}
  We've tried to address the concerns expressed in a thoughtful way, respectful,
  always, of the history and traditions of the game, and of player concerns. Our
  guiding star in thinking about changes to the game has always been our fans.
  What do our fans want to see on the field? We've conducted thorough and ongoing
  research with our fans, and certain things are really clear. Number 1, fans want
  games with better pace. Two, fans want more action, more balls in play. And three,
  fans want to see more of the athleticism of our great players.
\end{quote}

Also, keep in mind that these rule changes have been thoroughly tested and refined
in the Minor Leagues for many years so it's not like it was a split decision to
implement them in the Majors. Each of the rules were actually tested in about 8,000
Minor League games \citep{Castrovince}.

The next section is a Literature Review that details some relevant works on the
topic, but it is important to know that the data is so new that the impact of the
rule changes on various aspects of baseball performance metrics and strategies 
is not well understood yet. This study uses statistical methods to evaluate the
effects of the rule changes on game duration, scoring, hitting, pitching, fielding,
and baserunning. For instance, a major baseball statistic that is analyzed is
stolen bases (SB) which we should see an increase of across the league due to the
rule changes. Bigger bases mean slightly less distance traveled to get the
SB, and the pitch timer means that baserunners can time up the pitchers easier
and know when to start stealing. We compare the 2023 season data with the previous
seasons using descriptive statistics, hypothesis testing, and regression analysis.

This analysis is important as it holds the potential to inform future rule changes,
offering empirical insights that can guide decision-makers in shaping the game's
future. The data and findings generated by this research may serve as a valuable
resource for MLB executives, team managers, and players in adapting to the new
rule landscape. Equally crucial, fans and enthusiasts will gain deeper insights
into the impact of these changes on the sport they love, providing context to
their experiences as spectators.

% roadmap
The rest of the paper is organized as follows.
A literature review of relevant works will be examined in Section~\ref{sec:lit}.
The data used in the study will be presented in Section~\ref{sec:data}.
The methods are described in Section~\ref{sec:meth}.
The results are reported in Section~\ref{sec:resu}.
A discussion concludes in Section~\ref{sec:disc}.


\section{Literature Review}
\label{sec:lit}

As stated previously, there has been no scholarly literature yet on the effects
of the 2023 MLB rule changes (as of fall 2023), since it is so new. However,
there has been research done already on how a certain rule change/changes
affected a sport. For example, as referenced above, \citet{Cooley} investigated
the impact of the Designated Hitter (DH) in the MLB, a rule change that let a batter
hit for a pitcher in the lineup, instituted in 1973. They analyzed how the
implementation of the DH impacted certain metrics like attendance, complete games
by pitchers (CG), and total runs scored. They did so using various statistical
methods such as correlation analysis, regression (particularly r-squared analysis),
and compound average growth rates. They concluded that the DH led to more offensive
production, a decrease in total complete games by pitchers, and had no significant
effect on overall fan attendance at games.

However, we do not need to just look at Major League Baseball to see
the effects of some rule changes in sports. For instance, \citet{Madera}
discusses the impact of certain rule changes on game-related statistics in
professional men's water polo matches. They did so by calculating means and
standard deviations for certain game-related stats such as possession time,
total goals, etc., and running one-way ANOVA to test for significance in the
difference of these metrics pre and post rule changes. In a similar fashion,
\citet{Nourayi} analyzed the effects of the new defensive 3-second rule and
elimination of hand checking in the National Basketball Association (NBA), rule
changes aimed at improving the flow and pace of the game. This was done using
many different t-tests which resulted in finding that the total FGA (Field Goals
Attempted) and 3FGA (3-Point Field Goals Attempted) were significantly greater
in the post rule change period.

Also, we do not need to just look at rule changes specifically, we can explore
a ``before and after analysis'' of some other change made in sports. \citet{Price}
is a good illustration of this is, as they conducted an analysis of the COVID
``bubble'' in the NBA playoffs and how this affected home-court advantage.
Nine different z-tests were conducted to try to see if there was significance in
the difference in free-throw rate, winning percentage, etc. in the 2020 season
compared to previous years. This highly relates to our study, but we will instead
be testing the difference in say stolen bases for example, from the 2023 season
compared to previous years. \citet{Zhang} is another great example as they
focused on the effect of the video assistant referee (VAR) on referee's decisions
in the FIFA Women's World Cup. They used a Mann-Whitney U test to compare differences
in game-related variables such as playing time, penalties, and goals, with and
without VAR.


\section{Data}
\label{sec:data}

The data used for this analysis was collected from \href{https://baseballsavant.mlb.com/}{Baseball Savant},
a comprehensive and interactive platform that contains advanced statistics and
data related to Major League Baseball (MLB). The main variables of interest
are Stolen Bases (SB), Batting average (BA), Strike Outs (SO), and Batting
Average on Balls in Play (BABIP), among some others. These are game-related
baseball statistics that will be analyzed similar to those used in the literature
above, like we saw with Free Throw Rate in the NBA, for example. These variables
were selected based on domain knowledge and intuition about what metrics may be
affected most by the new MLB rule changes. We expect to see Stolen Bases increase
across the league, due to the pitch clock and pitcher disengagement limits. Batting
Average, Batting Average on Balls in Play, and Slugging (SLG) may be affected by
the limit of the defensive shift, as this rule change may increase the likelihood
of a hit. In a similar vein, Strike Outs and Walks (BB) may see an impact as both
pitchers and batters adjust to the pitch clock. Other variables included in the
study:

\begin{itemize}
  \item Ground Into Double Play (GIDP)
  \item Hits on Ground Balls (Hit-GB)
  \item Expected Batting Average (xBA)
\end{itemize}

We expect these variables will be affected by the limit of the defensive shift.
Also, pace of play (game time) data was collected from \href{https://www.baseball-reference.com/}{Baseball Reference}.
Each row of this dataset represents a single season from 2023 dating back to 1998
(26 observations total), containing the Average Game Time (in hours) for that season.
Intuitively, we should expect this variable to have a significant decrease for the
2023 season, due to the introduction of the pitch clock making the games go faster.
Ideally, we would have used datasets containing the game times for \underline{each}
game in both seasons, however this was not publicly available.

The main data used though was collected using the Custom Leaderboard feature
on Baseball Savant, where each variable above was added to the dataset, with each
row representing a single player's overall statistics. The data was then filtered
to only include the 2023 season, and only include players with a minimum 100 Plate Appearances (PA)
that season to limit outliers (100 min. PA is a popular threshold used for 
baseball player analysis). This gives us 461 observations for the 2023 season,
again with each row being a unique batter. The same steps were then applied to
create a comparison dataset now filtering for the 2022 season, which has 469
observations total. This gives us a baseline to compare the variables in the
season where the rule changes were in effect (2023), to the previous season where
they were not (2022). There are no missing or null values in each dataset.

Different data formats were considered, such as using team-wide statistics
for each season instead of player-wide, where the dataset would be 30 rows (one
for each team). Using a combination of previous seasons data for comparison (not
just 2022) was considered also, but we settled on the data we used as it seemed
the most straightforward for our analysis and limited potential confounding
factors. Data from complete game logs for each game of the 2023 season would have
been an option, however this data is not yet publicly available. More on this topic
is discussed in Section~\ref{sec:disc}.

One of the main reasons commissioner Rob Manfred enacted the pitch clock ``rule''
was to speed up the games and improve the pace of play to make them more engaging
for the fans, and preliminary analysis shows this worked. Notice in Figure~\ref{fig:Fig1}
the average length of games for the 2023 season was the shortest in the past 25 years.
There is some season-to-season variation but the highlighted yellow bar on the right
(representing the 2023 season) seems significantly lower than the rest. We will
test if the difference in average game length between 2023 and 2022 is statistically
significant in Section~\ref{sec:meth}.

\begin{figure}[tbp]
  \centering
  \includegraphics{barplot_glength}
  \caption{Average Game Length in hours for all MLB games from 1998 to 2023.}
  \label{fig:Fig1}
\end{figure}

To visually compare 2023 baseball statistics versus those in 2022, two histograms
were created for each variable with 2023's data overlaid on 2022's. Figure~\ref{fig:Fig2}
illustrates an example of this for the Stolen Bases variable, as we think this
metric may be most affected by the rule changes, as stated previously. Both
histograms exhibit fairly similar distributions; however, in 2023, there is a
noticeable increase in the number of players with 20 or more stolen bases. This
observation supports our intuition that the rule changes contributed to an uptick
in this variable. Side by side boxplots for 2022 versus 2023 were also created for
each variable as shown in Figure~\ref{fig:Fig3}. The Stolen Bases statistic again
sticks out, having a greater median and more high outliers in 2023. Expected Batting Average (XBA)
and Slugging Percentage (SLG Percent) both look intriguing as well. As said above
with Game Length, the statistical significance of the difference in these statistics
from 2022 to 2023 will be tested in Section~\ref{sec:meth}.

\begin{figure}[tbp]
  \centering
  \includegraphics{sb_hist}
  \caption{Comparison of Stolen Bases between the 2023 and 2022 MLB Seasons. The
  histograms illustrate the distribution of stolen base counts for players in each
  season, with blue representing 2023 and orange representing 2022.}
  \label{fig:Fig2}
\end{figure}

\begin{figure}[tbp]
  \centering
  \includegraphics{boxplots}
  \caption{Matrix of boxplots illustrating the evolution of baseball statistics
  from 2022 to 2023, encompassing strikeout rates, walk rates, batting averages,
  slugging percentages, BABIP, total stolen bases, GIDP, ground ball hits, and xBA.}
  \label{fig:Fig3}
\end{figure}


\section{Methods}
\label{sec:meth}

The introduction of these rule changes in the MLB provides a new and exciting
opportunity to study how they impacted various aspects of the game. This study
mainly focuses on comparing baseball statistics from the 2023 season versus the
2022 season to limit potential confounding factors. If we instead compared the
post-rule change season to say the last 10 MLB seasons, observed differences
may not be solely due to the 2023 rule changes, but rather from the effects of
drastic changes in the game between now and then. For instance, there were other
rule changes made in the last 10 years, such as the National League (NL) adopting
the designated hitter in 2022, relief pitchers having to pitch to at least 3 batters,
etc., that impacted statistics around the league. The 2020 season was also
drastically different due to COVID-19, with less games being played and no fans in
the stands. The ``juiced balls'' conspiracy was also running rampant in the last
decade, with evidence to show that in some seasons, players were playing with
baseballs that traveled further and harder off the bat. Offensive and defensive
philosophy has also seen historical changes, with defensive shifting becoming
more and more prevalent and teams diving more and more into the ``home run or
bust''/launch angle theory with batting. However, 2022 and 2023 were fairly similar
in nature all things considered with the obvious change being the new rules implemented,
which we will test to see how they affected statistics in 2023.

Comparisons between the 2023 and 2022 seasons were made on offensive metrics such
as batting average, stolen bases, and slugging percentage, along with game length.
Comparing the difference in these metrics between seasons will reveal valuable insights
into understanding how the rule changes impacted the game.

The ten following specific research questions will test the effects of the 2023
MLB rule changes on the game of baseball:

\begin{enumerate}
  \def\labelenumi{\arabic{enumi}.}
  \item
    Is the average game length shorter in 2023 compared to the previous 25 years?
  \item
    Are players stealing more bases in 2023 than they did in 2022?
  \item
    Is the strikeout rate different in 2023 compared to 2022?
  \item
    Is the walk rate different in 2023 compared to 2022?
  \item
    Is the batting average different in 2023 compared to 2022?
  \item
    Is the slugging percentage different in 2023 compared to 2022?
  \item
    Is the batting average on balls in play different in 2023 compared to 2022?
  \item 
    Is the expected batting average different in 2023 compared to 2022?
  \item
    Is the number of hits on ground balls greater in 2023 compared to 2022?
  \item
    Is the number of ground balls into double plays different in 2023 compared to 2022?
\end{enumerate}

All ten questions can be tested using a standard two-sample comparison with the \(t\)-test.
The \(z\)-test was considered as done in \citet{Price}, but we do not know the population standard deviation
in this case. The \(t\)-test assumes the data is normally distributed so to account
for this, we also conducted the nonparametric, distribution free, Mann Whitney U
Test (also known as the Wilcoxon Rank Sum Test) to be sure of our results. 

Also, notice how some of these research questions call for a one-sided hypothesis test,
like the number of hits on ground balls, because we intuitively expect this number
to be greater in 2023 than in 2022 due to the limit of the defensive shift. Other
metrics that we were not sure about whether they would increase or decrease were
tested using a two-sided hypothesis test, simply testing for a significance in the
overall difference from season to season. Other methods such as regression were
considered for this study as well, but we are primarily interested in comparing
means and assessing the effect of a single variable, so a \(t\)-test is more appropriate
and straightforward.


\section{Results}
\label{sec:resu}

Table~\ref{tab:table1} summarizes the p-values for both the \(t\)-tests and Wilcoxon's
tests for the research questions / hypotheses detailed in Section~\ref{sec:meth}.
The difference in game length test was excluded from this table as this tested
2023 versus 1998-2022 rather than just 2022, due to the data restrictions. Also,
point estimates for each test were provided, one for the 2023 group and one for the
2022 group.

At the top of the table we see that the mean number of stolen bases for each player
in 2023 is 7.226, and 4.994 in 2022 (these are the ``point estimates'') for the tests.
This difference is statistically significant for both the \(t\)-test and Wilcoxon's test
at $\alpha$ = 0.05, the significance level used for all of the tests in the table,
as the p-values are 0.0000 (extremely small, rounded for presentation purposes) and
0.0011 respectively. Adjusting the p-values to account for multiple tests using the
Bonferroni correction also shows significance for both of these tests for stolen bases.
This gives us ample evidence to conclude that the mean number of stolen bases in 2023
significantly increased from 2022.

On the other hand, the number of strikeouts did not seem to significantly change,
as its \(t\)-test p-value is 0.1596. The Wilcoxon's test p-value is extremely small
however, so the two tests are conflicting in their conclusions. Once we look at the
95\% confidence interval (CI) of \((-1.46, 8.89)\) though, we see that it contains zero,
meaning we cannot say the difference in the number of strikeouts year to year is significant.
Moving down the table, other hypotheses that were not significant were Walks, BABIP,
Hits on Ground Balls, and Ground into Double Play (GIDP) for both tests at the
5\% significance level. Hits on Ground Balls was particularly surprising, as
we expected that with no defensive shift anymore, more ground balls would become
base hits.

The difference in Batting Average, Slugging Percentage, and Expected Batting Average (xBA)
from 2022 to 2023 were all statistically significant for both the \(t\)-test and
Wilcoxon's Rank Sum test as their p-values were all less than $\alpha$ = 0.05. However,
using the Bonferroni correction, the adjusted p-values for the Batting Average
tests no longer were significant. So now there is only enough evidence to support
that the difference from 2022 to 2023 in Slugging Percentage and xBA is significant.

Also, for the game time data not shown in Table~\ref{tab:table1}, the average game
length from 1998 to 2022 was 2.92 hours (2 hrs 55 mins), compared to only 2.65 hours
(2 hrs 39 mins) in 2023. The \(t\)-test and Wilcoxon's test for the difference in
these values yielded two extremely small p-values, so we have enough evidence to 
conclude that the average game length is significantly shorter in 2023 compared to
that in the previous 25 years. This was expected, as we assumed the pitch clock would
speed up the game considerably.

\begin{table}[tbp]
  \caption{The results from the 9 baseball statistics tests}
  \label{tab:table1}
  \centering
  \begin{tabular}[t]{lccccc}
    \toprule
    & 2023 & 2022 & \multicolumn{2}{c}{P-value}\\
    \cmidrule(lr){4-5}
    &          &                & \(t\)-test & Wilcoxon's\\
    \midrule
    Stolen Bases & 7.226 & 4.994 & 0.0000 & 0.0011 \\ 
    Strikeouts & 85.445 & 81.731 & 0.1596 & 0.0000 \\ 
    Walks & 32.876 & 30.279 & 0.0599 & 0.0791 \\ 
    Batting Avg & 0.243 & 0.237 & 0.0274 & 0.0154 \\ 
    Slugging \% & 0.401 & 0.384 & 0.0008 & 0.0007 \\ 
    BABIP & 0.293 & 0.288 & 0.0793 & 0.0695 \\ 
    xBA & 0.241 & 0.235 & 0.0006 & 0.0005 \\ 
    Hits on GB & 27.345 & 26.399 & 0.3757 & 0.3212 \\ 
    GIDP & 7.232 & 6.925 & 0.3523 & 0.5401 \\ 
    \bottomrule
  \end{tabular}
\end{table}


\section{Discussion}
\label{sec:disc}

This study provides valuable insights into the immediate effects of the 2023 MLB
rule changes on various baseball statistics. Notably, there is robust evidence
indicating a significant increase in the mean number of stolen bases during the
2023 season. The introduction of a pitch clock and pitcher disengagement limits
appears to have influenced baserunning strategies, fostering a more aggressive
approach on the base paths. Additionally, offensive performance metrics such as
slugging percentage and expected batting average demonstrated significant changes,
suggesting that hitters adapted their approaches to capitalize on the new game dynamics.

However, several limitations should be acknowledged. The study's focus on a single-season
analysis, comparing 2023 to the previous season (2022), provides a snapshot of the
immediate impacts but falls short of capturing potential trends that may have been
happening over multiple seasons. External factors, such as individual player performance,
team strategies, and other influences like basic season to season variation, could have
contributed to observed differences, necessitating a more nuanced examination.
Furthermore, data constraints, particularly the availability of detailed game-level
data for certain metrics, limited the depth of the analysis.

In light of these limitations, future research directions are identified. Long-term
analyses spanning multiple seasons would provide a more comprehensive understanding
of the rule changes' lasting effects. Exploring player-level impacts and variations
across teams could reveal nuanced insights into how individual athletes and teams
adapted to the new rules. Additionally, a broader set of baseball metrics, including
pitcher statistics like Earned Run Average (ERA), and defensive statistics like
Outs Above Average (OAA), could contribute to a more holistic assessment of the rule
changes' impact. Further research assessing fan engagement, satisfaction, and attendance
would enhance our understanding of how these rule changes are perceived by the public.

All in all, while this study sheds light on the immediate impacts, the complex and
dynamic nature of baseball necessitates continuous monitoring and in-depth
analysis to grasp the different effects of rule changes on the sport. The findings
presented here serve as a great starting point for a more nuanced exploration of the
evolving landscape of Major League Baseball.


\bibliographystyle{chicago}
\bibliography{refs.bib}


\end{document}